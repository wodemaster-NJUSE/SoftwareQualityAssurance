\documentclass{article}

\usepackage{amsmath} % Used for mathematical formulas
\usepackage{graphicx} % Used for inserting images
\usepackage{lipsum} % Used for generating placeholder text
\usepackage{ctex} % Imported the ctex package to support Chinese
% \usepackage{titlesec} % Imported the titlesec package for customizing title styles
% \usepackage{fontspec} % Used for setting Chinese fonts

% \setmainfont{SimSun} % Set the Chinese font to SimSun (宋体 system font)

\usepackage{listings}
\usepackage{color}
\usepackage{float}
\title{基于LLM的选课管理系统项目开发方案}
\author{程智镝,汪天佑,陈凌,刘辉}
\date{\today}

\begin{document}
\maketitle

\textbf{前端技术栈:}
\begin{itemize}
    \item \textbf{UI 库}从 Bootstrap 改为 Ant Design Vue 以更好地匹配现代前端开发需求。
\end{itemize}

\textbf{后端技术栈:}
\begin{itemize}
    \item \textbf{消息队列}从通用消息队列改为 RabbitMQ 以增强异步任务处理能力。
    \item \textbf{测试工具}使用 JUnit 和 Mockito 进行单元测试。
\end{itemize}

\textbf{日志和监控:}
\begin{itemize}
    \item \textbf{日志库}从 Log4j 改为 Log4j2 以利用其更好的性能和特性。
\end{itemize}

\textbf{架构设计:}
\begin{itemize}
    \item 提供更详细的前端组件和后端组件描述。
\end{itemize}

\textbf{开发计划:}
\begin{itemize}
    \item 保持原计划,但对每周的具体任务进行了更详细的描述。
\end{itemize}

\textbf{项目风险管理:}
\begin{itemize}
    \item 增加了技术、管理和资源风险的详细描述,并提供了更具体的应对策略。
\end{itemize}

\textbf{功能特点}
\begin{itemize}
        \item 在用户信息收集中,增加表单验证和格式校验,保证收集数据的有效性和正确性.
        \item 在用户登录中, 增加多因素认证手段以增加登录操作的安全性和便利性
        \item 在用户权限管理中, 新增动态权限管理的要求, 权限的修改需要可以实时生效.
        \item 在实时反馈中,系统的智能化选课建议新增对学生课内生活与课外生活的平衡,确保学生可以在学习和生活间达到平衡,这更符合选课系统的理念.
\end{itemize}

\textbf{质量保证计划}
\begin{itemize}
	\item 增加了设定质量度量指标的计划,以追踪和评估项目的质量,确保质量目标的实现。
	\item 增加了完善项目相关文档的计划,确保项目文档的完整性和准确性,以提供高质量的文档支持。
\end{itemize}

\textbf{日常维护}
\begin{itemize}
	\item 增加了定期维护系统的文档的计划,保持系统文档处于最新状态,确保文档的准确性和完整性。
\end{itemize}

\textbf{升级计划}
\begin{itemize}
	\item 增加了制定升级后的评估和反馈机制的计划,可以及时收集用户反馈,评估升级的效果,并继续进行相应的改进。
\end{itemize}


\section{检查后的修改}
针对检查过程中量化不足,缺乏足够估算问题,我们做了以下修改
\subsection{需求分析部分}
对于各个功能需求,我们进行了代码量和人时量的评估
\subsection{开发计划部分}
添加了里程碑,并且细化了项目开发时间规划,增添日程计划
\subsection{功能特点部分}
添加了各个功能的代码当量和所需人时的评估,并以表格形式列出
\subsection{质量保证部分}
添加了对A/FR指标的具体计算
\subsection{预算与成本部分}
添加了量化的成本图表,展示了第一年与第二年的成本统计

\end{document}
