\documentclass{article}

\usepackage{amsmath} % Used for mathematical formulas
\usepackage{graphicx} % Used for inserting images
\usepackage{lipsum} % Used for generating placeholder text
\usepackage{ctex} % Imported the ctex package to support Chinese
% \usepackage{titlesec} % Imported the titlesec package for customizing title styles
% \usepackage{fontspec} % Used for setting Chinese fonts

% \setmainfont{SimSun} % Set the Chinese font to SimSun (宋体 system font)

\usepackage{listings}
\usepackage{color}
\usepackage{float}
\title{基于LLM的选课管理系统项目开发方案}
\author{程智镝,汪天佑,陈凌,刘辉}
\date{\today}

\begin{document}
\maketitle

\textbf{前端技术栈:}
\begin{itemize}
    \item \textbf{UI 库}从 Bootstrap 改为 Ant Design Vue 以更好地匹配现代前端开发需求。
\end{itemize}

\textbf{后端技术栈:}
\begin{itemize}
    \item \textbf{消息队列}从通用消息队列改为 RabbitMQ 以增强异步任务处理能力。
    \item \textbf{测试工具}推荐使用 JUnit 和 Mockito 进行单元测试。
\end{itemize}

\textbf{日志和监控:}
\begin{itemize}
    \item \textbf{日志库}从 Log4j 改为 Log4j2 以利用其更好的性能和特性。
\end{itemize}

\textbf{架构设计:}
\begin{itemize}
    \item 提供更详细的前端组件和后端组件描述。
\end{itemize}

\textbf{开发计划:}
\begin{itemize}
    \item 保持原计划,但对每周的具体任务进行了更详细的描述。
\end{itemize}

\textbf{项目风险管理:}
\begin{itemize}
    \item 增加了技术、管理和资源风险的详细描述,并提供了更具体的应对策略。
\end{itemize}


\end{document}