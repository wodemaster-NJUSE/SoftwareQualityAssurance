\documentclass{article}

\usepackage{amsmath} % Used for mathematical formulas
\usepackage{graphicx} % Used for inserting images
\usepackage{lipsum} % Used for generating placeholder text
\usepackage{ctex} % Imported the ctex package to support Chinese
% \usepackage{titlesec} % Imported the titlesec package for customizing title styles
% \usepackage{fontspec} % Used for setting Chinese fonts

% \setmainfont{SimSun} % Set the Chinese font to SimSun (宋体 system font)

\usepackage{listings}
\usepackage{color}
\usepackage{float}

\title{基于LLM的选课管理系统项目开发方案}
\author{程智镝}
\date{\today}

\begin{document}
\maketitle

% 添加目录
\tableofcontents

\newpage

\section{项目概述}

\section{关键问题}
\subsection{项目执行问题}
TODO:对应作业要求中的关键问题
\subsection{团队能力和信任问题}
\subsection{风险管理}
\subsection{用户体验和反馈问题}
\subsection{数据安全和隐私问题}
\subsection{持续维护和升级问题}

\section{需求分析}
\subsection{需求规格说明描述}
\subsection{需求中的主要特点和挑战}
\subsection{系统的稳定性、性能、安全性和可扩展性}



\section{解决方案概要}
\subsection{技术栈选择}
\textbf{前端技术栈:}

\begin{itemize}
  \item 前端框架:使用流行的前端框架,Vue.js,以构建用户友好的界面和提供丰富的交互体
  验。
  \item HTML/CSS: 使用 HTML 和 CSS 来设计和布局网页,确保页面加载速度快且兼容性好。
  \item JavaScript: 使用 JavaScript 编写客户端逻辑,实现用户操作的动态效果和交互功能。
  \item UI 库:使用 UI 库,Bootstrap,以便快速构建具有一致性和响应性的界面组件。
\end{itemize}

\textbf{后端技术栈:}

\begin{itemize}
  \item 后端框架:使用适合 Web 应用程序的后端框架,SpringBoot。
  \item 数据库:选择适当的数据库管理系统,MySQL主从复制或分布式数据库系统 以存储和管理学生选课数据。
  \item API 和数据格式:使用 RESTful API 来处理数据传输和交互,并采用 JSON 或 XML 等标
  准数据格式。
  \item 身份验证和安全性:使用身份验证库和安全性框架来确保用户数据和系统的安全性。
\end{itemize}

\textbf{其他技术组件:}

\begin{itemize}
  \item 服务器:部署应用程序的服务器,可以选择云托管服务提供商阿里云。搭建多个相同配置的后端服务器以应对高并发请求
  \item 消息队列: 使用消息队列来处理邮件通知和异步任务,提高系统的可靠性和性能。
  \item 缓存:使用缓存技术 Redis,以提高数据检索和响应速度。
  \item 负载均衡:负责将传入的请求分发到多个后端服务器,实现负载均衡和高可用,采用软件负载均衡器Nginx实现
  \item 测试工具: 选择适当的测试工具和框架,确保代码的质量和稳定性。
\end{itemize}

\subsection{日志和监控}

\textbf{日志记录:}
\begin{itemize}
  \item 选择日志库:选择适当的日志库 Log4j 以便在应用程序中记录日志。
  \item 设置日志级别:配置不同级别的日志记录,如调试、信息、警告和错误,以便根据需要过
  滤和查看日志。
  \item 日志格式: 定义日志格式,包括时间戳、日志级别、消息内容以及源代码位置等信息。
  \item 日志存储: 将日志存储在可访问的位置,例如本地文件、数据库或日志管理平台。
\end{itemize}

\textbf{监控工具:}
\begin{itemize}
  \item 应用性能监控(APM):使用 APM 工具 New Relic,来监测应用程序的性能,包括响应时
  间、事务追踪和错误追踪。
  \item 基础设施监控:使用基础设施监控工具 Prometheus,监测服务器资源利用、网络流量和数
  据库性能。
  \item 日志管理平台:集成日志管理平台 ELK Stack,用于集中存储、搜索和可视化日志数据。
\end{itemize}

\textbf{实时警报:}
\begin{itemize}
  \item 设置警报规则:配置警报规则,以便在关键事件发生或性能达到临界值时触发警报
  \item 通知方式: 集成通知方式,如电子邮件、短信、Slack 消息等,以便在触发警报时及时通
  知相关人员。
\end{itemize}

\textbf{日志分析和仪表板:}
\begin{itemize}
  \item 创建仪表板:使用仪表板工具 Grafana,可视化监控数据和日志记录,以便实时查看系统
  状态。
  \item 日志分析: 使用搜索和查询功能来分析日志数据,以便排查问题、监测趋势和提取有用的
  信息。
\end{itemize}

\textbf{定期审查和优化:}
\begin{itemize}
  \item 定期审查: 定期审查监控数据、日志记录和警报历史,以发现潜在问题和性能瓶颈。
  \item 性能优化: 根据监控数据的反馈,优化系统性能,可能包括代码优化、资源扩展和配置调
  整。
\end{itemize}


\subsection{架构设计}
\textbf{系统整体架构概述}
选课系统采用了典型的三层架构,包括前端、后端(包含LLM)和数据库层。这种架构将系统的不
同部分清晰地分离,以实现模块化、可扩展和易维护的设计。


\textbf{前端组件}
前端是用户与系统互动的界面,负责呈现用户界面、处理用户输入和与后端通信。前端组
件包括以下关键特点:
\begin{itemize}
  \item 用户界面(UI): 使用现代前端框架构建用户友好的界面,以提供直观的用户体验。
  \item 用户认证和授权: 实施用户登录和身份验证机制,根据用户角色授权不同的操作权限。
  \item 学生选课和LLM管理:前端负责展示
  \item 通知和警报: 通过前端界面向用户发送通知和警报,包括考试密码和考试结果的邮件通
  知。
  \item 性能优化: 前端应具备性能优化策略,包括资源缓存、异步加载和响应式设计,以确保系
  统在不同设备上表现良好。
\end{itemize}

\textbf{后端组件:}
\begin{itemize}
  \item 应用服务器: 使用后端框架构建应用服务器,处理前端请求并执行业务逻辑
  \item 数据库管理:使用适当的数据库系统来存学生选、大模型会话信息、课程信息和用户信息。
  \item API 接口: 提供 RESTful API 接口,用于前端和后端之间的数据传输和交互,包括课程计划管理、选课管理、大语言会话信息处理等。
  \item 身份验证和安全性: 实施用户身份验证和授权,保护用户数据和系统安全。
  \item 性能优化和缓存: 优化后端代码以提高性能,使用缓存来减轻数据库负载。
  \item 消息队列: 集成消息队列,以异步处理邮件通知和其他后台任务。
\end{itemize}

\subsection{数据库设计}
\begin{itemize}
    \item 数据库管理系统: 使用合适的关系型或非关系型数据库管理系统来存储数据。
    \item 数据模型设计: 设计合适的数据库表结构,以支持数据的高效检索和存储。
    \item 数据备份和恢复: 实施定期的数据备份策略,以确保数据的安全性和可用性。
  \end{itemize}


\section{开发计划}
\subsection{项目时间表}
\textbf{第1周:项目开发和集成}
\begin{itemize}
    \item 周一至周三:进行各个模块的开发,确保每个团队成员按照分好的工进行开发。
    \item 周四至周五:开始模块集成,确保各个模块能够协同工作,处理接口的问题。
\end{itemize}

\textbf{第2周:测试和优化}
\begin{itemize}
    \item 周一至周三:进行考试系统整体测试,包括功能测试、性能测试、安全测试等。
    \item 周四:修复测试中发现的问题,进行考试系统性能优化。
    \item 周五:完成剩余的优化工作,确保考试系统在各种条件下都能够正常运行。
\end{itemize}

\textbf{第3周:上线前准备}
\begin{itemize}
    \item 周一至周三:部署系统到预上线环境,进行最后的测试和调优。
    \item 周四:准备上线所需的文档、备份和监控系统。
    \item 周五:上线发布项目,进行线上监控和备份。
\end{itemize}

\textbf{第4周:维护和反馈}
\begin{itemize}
    \item 周一至周三:监控选课系统,处理可能出现的问题和bug。
    \item 周四:与用户(考生,老师,管理员等)进行反馈交流,收集用户意见和建议,做好用户满意度调查。
    \item 周五:根据用户反馈,进行必要的调整和修复。
\end{itemize}
\subsection{项目风险管理}
\textbf{技术风险:}
\begin{itemize}
    \item \textbf{风险:} 选择的技术栈可能不够成熟或无法满足系统需求。
    \item \textbf{应对策略:} 在项目前期进行技术评估,验证所选技术的适用性。建立备选方案以备不时之需。
\end{itemize}

\textbf{安全风险:}
\begin{itemize}
    \item \textbf{风险:} 数据泄露、漏洞和未经授权的访问可能导致系统安全问题。
    \item \textbf{应对策略:} 实施强大的身份验证和授权机制,定期进行安全审计和漏洞扫描,及时修复发现的漏洞。
\end{itemize}

\textbf{人员风险:}
\begin{itemize}
    \item \textbf{风险:} 项目团队中的关键成员可能离开或出现能力不足的问题。
    \item \textbf{应对策略:} 确保团队有足够的人员资源,建立知识共享和培训计划,减轻对个别成员的依赖。
\end{itemize}

\textbf{范围风险:}
\begin{itemize}
    \item \textbf{风险:} 需求变更或误解可能导致范围蔓延。
    \item \textbf{应对策略:} 建立严格的变更控制流程,确保每项需求变更都经过评审和批准。与利益相关者进行积极的沟通。
\end{itemize}

\textbf{时间风险:}
\begin{itemize}
    \item \textbf{风险:} 项目进度可能受到延误,导致无法按计划上线。
    \item \textbf{应对策略:} 制定详细的项目计划,设定里程碑并进行定期的进度追踪。提前识别并解决延误问题。
\end{itemize}

\textbf{资源风险:}
\begin{itemize}
    \item \textbf{风险:} 资金、人力和硬件资源可能不足。
    \item \textbf{应对策略:} 在项目启动前进行资源规划,与高层管理层协商项目预算,确保有足够的资源支持项目需求。
\end{itemize}

\textbf{集成和性能风险:}
\begin{itemize}
    \item \textbf{风险:} 不同组件之间的集成问题可能导致系统性能下降或故障。
    \item \textbf{应对策略:} 进行持续的集成测试,确保各个组件协同工作,同时进行性能测试以发现并解决性能问题。
\end{itemize}

\textbf{管理风险:}
\begin{itemize}
    \item \textbf{风险:} 不良的项目管理实践可能导致项目控制失效。
    \item \textbf{应对策略:} 使用有效的项目管理工具和方法,建立明确的沟通渠道,定期审查项目进展。
\end{itemize}
\section{功能特点}

\subsection{登录和用户管理}
\subsubsection{用户注册}
\begin{itemize}
        \item 用户信息收集: 在注册页面,要求用户提供必要的个人信息,如姓名、学号、邮箱地址等。确保用户信息的完整性和准确性。
        \item 用户角色选择: 用户在注册时选择其角色,通常分为学生、教师和管理员。每个角色有不同的权限。
        \item 密码安全: 强制要求用户创建强密码,包括字母、数字和特殊字符,并确保密码加密存储。
        \item 邮箱验证: 发送验证邮件到提供的邮箱地址,包括一个唯一的确认链接。用户需要点击确认链接以完成注册。
        \item 防止重复注册: 确保同一邮箱地址或学号不能多次注册。   
\end{itemize}

\subsubsection{用户登录}
\begin{itemize}
        \item 用户名密码登录: 用户输入已注册的邮箱地址和密码,系统验证这些信息后允许用户登录。
        \item 记住我: 提供“记住我”选项,以便用户在下次访问时免除重新登录。
        \item 单点登录(可选): 如果系统需要与其他系统集成,可以实施单点登录(SSO)以简化用户登录流程。 
\end{itemize}
\subsubsection{角色权限管理}
\begin{itemize}
        \item 角色定义: 定义不同角色的权限,如学生、教师和管理员。每个角色有不同级别的访问和操作权限。
        \item 审计日志: 记录用户的登录和操作,以便审计角色权限的使用情况。
        \item 动态权限管理: 允许管理员根据需要更改角色权限,以反映实际情况。
        \item 用户角色切换(如果适用): 如果用户具有多个角色,例如一个教师也可以是学生,实现用户角色切换功能。
        \item 密码重置和账户锁定: 提供密码重置功能,以及在多次登录失败后锁定账户以提高安全性。
        \item 权限分配: 将权限与角色关联,以便更轻松地为用户分配权限。 
        \item 角色验证: 在系统的各个部分使用角色验证,确保用户只能执行其具有权限的操作。
\end{itemize}
\subsection{课程管理}
\subsubsection{课程导入}
\begin{itemize}
        \item 灵活性: 系统应具有足够的灵活性,能够处理不同类型的课程,例如公选课,必修课,通识课等等
        \item 数据验证: 在课程导入过程中,进行数据验证,确保课程信息的正确性和完整性。例如,检查每个课程是否包含课程描述、老师等必要信息
\end{itemize}
\subsubsection{课程调整和停止}
\begin{itemize}
        \item 灵活性: 系统应具有足够的灵活性,可以在允许范围内让教师对课程进行任意修改
        \item 兜底限制: 应做好对课程修改的兜底限制,例如: 时间不能调制周末等
\end{itemize}

\subsection{选课策略设置}
\subsubsection{定制化课程参数设置}
\begin{itemize}
        \item 课时设置: 允许管理员或教师根据课程的需要来设定课程的课时
        \item 考试方式设置: 允许教师根据课程本身的特性和需求来设定课程考核的形式
        \item 选课方式设置: 允许教师根据课程特性设定课程的选课方式,例如:指选,抽签等
\end{itemize}
\subsubsection{学生名单导入}
\begin{itemize}
        \item 数据格式: 允许管理员或教师导入学生名单数据,数据格式可以是常见的 Excel 格式或CSV 格式。
        \item 数据验证: 在导入过程中,进行数据验证,以确保学生名单数据的准确性。例如,检查每个学生是否包括必要的信息,如姓名、学号、班级等。
        \item 批量导入: 支持批量导入,以便一次性导入多个学生名单,减少手动输入的工作量。
        \item 重复数据处理: 处理可能存在的重复数据或重复学生记录,以避免重复导入。
\end{itemize}

\subsection{选课流程}
\subsubsection{自主课程选择}
\begin{itemize}
        \item 选课时间设定: 管理员可以为不同类型的课程设定不同的选课时间
        \item 课程呈现: 在选课开始后,系统应呈现课程列表给同学,以便他们选择课程。
        \item 课程信息预览: 提供教师和学生查看课程信息的功能,以确保选择时可以看见课程的内容。
\end{itemize}
 
\subsubsection{选课过程支持}
\begin{itemize}
        \item 备选课程展示: 学生应可以对课程加标注,学生可以进行备选课程列表快速选择
        \item 汇总页面: 系统应提供一个汇总页面,显示学生所有选择的课程。这样考生可以了解自己的选课情况。
        \item 超链接导航: 系统应提供超链接,允许学生在不同类型课程之间轻松跳转,并返回到汇总页面。
\end{itemize}

\subsubsection{防外挂功能}
\begin{itemize}
        \item 反脚本技术: 系统应该做好对学生操作的监控与限流,防止学生使用脚本选课
        \item 账号锁定: 如果选课期间出现可疑指标,管理员可以远程锁定学生账户,防止进一步作弊。
\end{itemize} 

\subsection{智能化体验}
\subsubsection{智能调配}
\begin{itemize}
\item 课程冲突检测: 系统可以自动检测学生所选择的课程是否存在时间上的冲突,并提供解决方案,如调整课程时间或寻找替代课程。
\item 课程平衡建议: 基于学生的学业规划和课程安排,系统可以智能地建议学生选择一定比例的核心课程、选修课程和兴趣课程,以保持学业平衡。
\end{itemize}

\subsubsection{实时反馈}
\begin{itemize}
\item 选课结果预测: 系统可以根据学生的选课情况和历史数据,预测学生最终选择的课程,并提供相应的反馈和建议。
\item 选课建议优化: 系统可以根据学生的反馈和实际选课结果,不断优化推荐算法,提供更加准确和个性化的选课建议。
\end{itemize}


\section{用户体验和界面设计}

\section{测试和质量保证}
\subsection{测试计划}
\section{部署和维护}
\subsection{部署计划}
\subsection{维护策略}

\section{预算和资源}
\subsection{硬件需求}
\subsection{软件需求}
\subsection{人力资源需求}

\section{推进计划}

\section{结论}
\end{document}
