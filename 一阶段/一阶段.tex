\documentclass{article}

\usepackage{amsmath} % Used for mathematical formulas
\usepackage{graphicx} % Used for inserting images
\usepackage{lipsum} % Used for generating placeholder text
\usepackage{ctex} % Imported the ctex package to support Chinese
% \usepackage{titlesec} % Imported the titlesec package for customizing title styles
% \usepackage{fontspec} % Used for setting Chinese fonts

% \setmainfont{SimSun} % Set the Chinese font to SimSun (宋体 system font)

\usepackage{listings}
\usepackage{color}
\usepackage{float}

\title{基于LLM的选课管理系统项目开发方案}
\author{程智镝}
\date{\today}

\begin{document}
\maketitle

% 添加目录
\tableofcontents

\newpage

\section{项目概述}

\section{关键问题}
\subsection{项目执行问题}
TODO:对应作业要求中的关键问题
\subsection{团队能力和信任问题}
\subsection{风险管理}
\subsection{用户体验和反馈问题}
\subsection{数据安全和隐私问题}
\subsection{持续维护和升级问题}
\section{需求分析}
\subsection{需求规格说明描述}
\subsection{需求中的主要特点和挑战}
\subsection{系统的稳定性、性能、安全性和可扩展性}
\section{解决方案概要}
\subsection{技术栈选择}
\subsection{日志和监控}
\subsection{架构设计}
\subsection{数据库设计}
\subsection{大模型设计}
\section{开发计划}
\subsection{项目时间表}
\section{功能特点}
\subsection{登录和用户管理}
\subsubsection{用户注册}
\begin{itemize}
        \item 用户信息收集: 在注册页面,要求用户提供必要的个人信息,如姓名、学号、邮箱地址等。确保用户信息的完整性和准确性。
        \item 用户角色选择: 用户在注册时选择其角色,通常分为学生、教师和管理员。每个角色有不同的权限。
        \item 密码安全: 强制要求用户创建强密码,包括字母、数字和特殊字符,并确保密码加密存储。
        \item 邮箱验证: 发送验证邮件到提供的邮箱地址,包括一个唯一的确认链接。用户需要点击确认链接以完成注册。
        \item 防止重复注册: 确保同一邮箱地址或学号不能多次注册。   
\end{itemize}

\subsubsection{用户登录}
\begin{itemize}
        \item 用户名密码登录: 用户输入已注册的邮箱地址和密码,系统验证这些信息后允许用户登录。
        \item 记住我: 提供“记住我”选项,以便用户在下次访问时免除重新登录。
        \item 单点登录(可选): 如果系统需要与其他系统集成,可以实施单点登录(SSO)以简化用户登录流程。 
\end{itemize}
\subsubsection{角色权限管理}
\begin{itemize}
        \item 角色定义: 定义不同角色的权限,如学生、教师和管理员。每个角色有不同级别的访问和操作权限。
        \item 审计日志: 记录用户的登录和操作,以便审计角色权限的使用情况。
        \item 动态权限管理: 允许管理员根据需要更改角色权限,以反映实际情况。
        \item 用户角色切换(如果适用): 如果用户具有多个角色,例如一个教师也可以是学生,实现用户角色切换功能。
        \item 密码重置和账户锁定: 提供密码重置功能,以及在多次登录失败后锁定账户以提高安全性。
        \item 权限分配: 将权限与角色关联,以便更轻松地为用户分配权限。 
        \item 角色验证: 在系统的各个部分使用角色验证,确保用户只能执行其具有权限的操作。
\end{itemize}
\subsection{课程管理}
\subsubsection{课程导入}
\begin{itemize}
        \item 灵活性: 系统应具有足够的灵活性,能够处理不同类型的课程,例如公选课,必修课,通识课等等
        \item 数据验证: 在课程导入过程中,进行数据验证,确保课程信息的正确性和完整性。例如,检查每个课程是否包含课程描述、老师等必要信息
\end{itemize}
\subsubsection{课程调整和停止}
\begin{itemize}
        \item 灵活性: 系统应具有足够的灵活性,可以在允许范围内让教师对课程进行任意修改
        \item 兜底限制: 应做好对课程修改的兜底限制,例如: 时间不能调制周末等
\end{itemize}

\subsection{选课策略设置}
\subsubsection{定制化课程参数设置}
\begin{itemize}
        \item 课时设置: 允许管理员或教师根据课程的需要来设定课程的课时
        \item 考试方式设置: 允许教师根据课程本身的特性和需求来设定课程考核的形式
        \item 选课方式设置: 允许教师根据课程特性设定课程的选课方式,例如:指选,抽签等
\end{itemize}
\subsubsection{学生名单导入}
\begin{itemize}
        \item 数据格式: 允许管理员或教师导入学生名单数据,数据格式可以是常见的 Excel 格式或CSV 格式。
        \item 数据验证: 在导入过程中,进行数据验证,以确保学生名单数据的准确性。例如,检查每个学生是否包括必要的信息,如姓名、学号、班级等。
        \item 批量导入: 支持批量导入,以便一次性导入多个学生名单,减少手动输入的工作量。
        \item 重复数据处理: 处理可能存在的重复数据或重复学生记录,以避免重复导入。
\end{itemize}

\subsection{选课流程}
\subsubsection{自主课程选择}
\begin{itemize}
        \item 选课时间设定: 管理员可以为不同类型的课程设定不同的选课时间
        \item 课程呈现: 在选课开始后,系统应呈现课程列表给同学,以便他们选择课程。
        \item 课程信息预览: 提供教师和学生查看课程信息的功能,以确保选择时可以看见课程的内容。
\end{itemize}
 
\subsubsection{选课过程支持}
\begin{itemize}
        \item 备选课程展示: 学生应可以对课程加标注,学生可以进行备选课程列表快速选择
        \item 汇总页面: 系统应提供一个汇总页面,显示学生所有选择的课程。这样考生可以了解自己的选课情况。
        \item 超链接导航: 系统应提供超链接,允许学生在不同类型课程之间轻松跳转,并返回到汇总页面。
\end{itemize}

\subsubsection{防外挂功能}
\begin{itemize}
        \item 反脚本技术: 系统应该做好对学生操作的监控与限流,防止学生使用脚本选课
        \item 账号锁定: 如果选课期间出现可疑指标,管理员可以远程锁定学生账户,防止进一步作弊。
\end{itemize} 

\subsection{智能化体验}
\subsubsection{智能调配}
\begin{itemize}
\item 课程冲突检测: 系统可以自动检测学生所选择的课程是否存在时间上的冲突,并提供解决方案,如调整课程时间或寻找替代课程。
\item 课程平衡建议: 基于学生的学业规划和课程安排,系统可以智能地建议学生选择一定比例的核心课程、选修课程和兴趣课程,以保持学业平衡。
\end{itemize}

\subsubsection{实时反馈}
\begin{itemize}
\item 选课结果预测: 系统可以根据学生的选课情况和历史数据,预测学生最终选择的课程,并提供相应的反馈和建议。
\item 选课建议优化: 系统可以根据学生的反馈和实际选课结果,不断优化推荐算法,提供更加准确和个性化的选课建议。
\end{itemize}

\section{用户体验和界面设计}
\section{测试和质量保证}
\subsection{测试计划}
\section{部署和维护}
\subsection{部署计划}
\subsection{维护策略}
\section{预算和资源}
\subsection{硬件需求}
\subsection{软件需求}
\subsection{人力资源需求}
\section{推进计划}
\section{结论}
\end{document}
